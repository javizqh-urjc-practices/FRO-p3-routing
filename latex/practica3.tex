\documentclass[12pt, a4paper]{report}
\usepackage{graphicx} %LaTeX package to import graphics
\usepackage{enumitem}
\usepackage{geometry}
\usepackage{xcolor}
\usepackage{amsmath}
\geometry{lmargin=30mm}
\usepackage[export]{adjustbox}
\usepackage{titlesec}
\titleformat{\chapter}{\normalfont\huge}{\thechapter}{20pt}{\huge\bf}
\graphicspath{{images/}} %configuring the graphicx package
\title{Practica 3}
\author{Javier Izquierdo Hernández}
\date{\today}
\begin{document}
	\begin{titlepage}
		\centering
		{\includegraphics[width=0.3\textwidth]{logo}\par}
		\vspace{1cm}
		{\bfseries\LARGE Universidad Rey Juan Carlos \par}
		\vspace{1cm}
		{\scshape\Large E.T.S. Ingeniería de Telecomunicación \par}
		\vspace{3cm}
		{\scshape\Huge Fundamentos de redes de ordenadores \par}
		\vspace{3cm}
		{\itshape\Large Práctica 3 \par}
		\vfill
		{\Large Autor: \par}
		{\Large Javier Izquierdo Hernández \par}
		\vfill
		{\Large \today \par}
	\end{titlepage}

\newpage
\renewcommand{\contentsname}{Contenidos}
\tableofcontents
\newpage
\chapter{Configuración de tablas de encaminamiento con route}
Este escenario aplica una configuración asignando direcciones IP a todas las interfaces de las máquinas, excepto a pc1.
Esta configuración inicial está almacenada en el fichero
/etc/network/interfaces de cada una de las máquinas, tal y como se ha visto en la práctica anterior.
\newline

\includegraphics*[scale=0.15, center]{enunciado1}
\newline
Teniendo en cuenta que sólo están arrancadas pc1, pc2, pc3 y pc4, responde a las siguientes cuestiones:
\begin{enumerate}
	\item Escribe en pc1 la orden
	ping 127.0.0.1 . ¿Por qué se obtiene respuesta pese a no tener configurada pc1 una dirección IP? Utilizando
	route comprueba el contenido actual de la tabla de  encaminamiento ( routing ) de 	pc1.
	
	Porque esa dirección IP es la correspondiente a la dirección de loopback. La tabla de encaminamiento está vacía.
	
	\includegraphics*[width=128mm, scale=0.5, center]{ejercicio1}
	\item Modifica el fichero /etc/network/interfaces de pc1
	para que tenga una direccion IP acorde a la de la
	subred a la que esta conectado. (Nota: Deberas consultar previamente la mascara de la subred que tienen
	las otras maquinas conectadas a la misma subred que pc1). Reinicia la red en pc1 para que se aplique la
	configuracion que has escrito en el fichero /etc/network/interfaces.
	\includegraphics*[width=128mm, scale=0.5, center]{ejercicio2}
	
	La IP será 200.57.0.10, ya que la máscara es 255.255.255.0
	\item Comprueba con route el contenido actual de su table de encaminamiento. Veras como, tras asignar la
	direccion IP a su interfaz de red, se ha anadido automaticamente una entrada en la tabla. Con esta tabla
	actual pc1, ¿a que otras direcciones IP crees que pc1 podra enviar datagramas IP?
	
	Podrá enviar datagramas IP a 200.57.0.40 pc4.
	
	\item Dado que el resto de las maquinas encendidas tienen ya configurada una direccion IP, podras suponer facilmente cual es el contenido de su tabla de encaminamiento. Mira la tabla de encaminamiento de pc2 y pc4. ¿A que otras direcciones IP crees que esas maquinas podran enviar datagramas IP?
	
	Pc2 solo podrá enviar datagramas a sí mismo o a r2 y r3.
	Pc4 podrá enviar datagramas a pc1 o a r1 y r3.

	\item Intenta deducir cual crees que sera la tabla de encaminamiento de r3,dado que tiene dos interfaces con IP asignada. Compruebalo consultando su tabla. Con esta tabla de encaminamiento,  ¿crees que r3 puede enviar datagramas IP a pc1 y pc4? ¿Y a pc2?
	
	La tabla de encaminamiento de r3 será la de las 2 interfaces Ip con gateway de 0.0.0.0.
	
	R3 podrá enviar datagramas a pc1 y a pc4 porque están conectadas a la interfaz Ip a la que esta conectada r3.
	
	R3 tambíen podrá enviar un datagrama a pc2 porque tienen una conexión directa.
	
	\item Haz ping desde pc1 a pc4 y haz ping desde pc1 a la direccion r3(eth0). Ten en cuenta que no puedes utilizar los nombres pc1, pc4, etc. en el ping, sino que debes usar las direcciones IP correspondientes.  ¿Funcionan
	estos ping?  ¿Que entradas de las tablas de encaminamiento se consultan en cada caso?
	
	Los dos ping funcionan, y las entradas que se consultan son las correspondientes a 200.57.0.0 con máscara de 255.255.255.0.
	\item Haz un ping de pc1 a pc2 y haz un  ping de pc1 a la direccion r3(eth1). ¿Funcionan estos ping? ¿Por que?
	
	No funcionan, porque la tabla de encaminamiento de pc1 no tiene esa direcciones Ip
	\item Añade una ruta con el comando route en pc1 para que los datagramas IP que no sean para su propia subred los envie a traves del router r3.
	
	\includegraphics*[width=128mm, scale=0.5, center]{ejercicio3}
	\item Haz ahora ping desde pc1 a r3(eth1).  ¿Funciona este ping?  ¿Que entradas de las tablas de encaminamiento se consultan?
	
	Si funciona, se consulta la última entrada de la tabla de encaminamiento.
	\item Haz un ping de pc1 a pc2. ¿Por que crees que no funciona este ping?
	
	Porque pc2 no tiene configurada en la tabla de encaminamiento la subred de pc1, por lo que no puede enviar el mensaje de vuelta.
	\item Añade las rutas necesarias utilizando el comado route para que funcione un ping de pc1 a pc2 y un ping de pc4 a pc2.
	Ten en cuenta que podras utilizar, rutas de maquina, rutas de subred o ruta por defecto.
	
	
	\item ¿Crees que con la configuracion que has realizado funcionara un ping de pc2 pc1 a y un ping pc2 a pc4? Compruebalo.
	
	Si funciona.
	\item Antes de continuar espera unos 10 minutos despues de haber 
		ejecutado el ultimo ping del apartado anterior.
		Consulta el estado de las caches de ARP en los pcs y en el router hasta que esten vacias.
		\newline
		Arranca en pc4 un tcpdump para capturar trafico en su interfaz eth0, guardando la captura en el fichero
		\color{blue}p3-a-01.cap.
		\newline
		\color{black}
		Ejecuta en pc1 un ping a pc4 que envie solo 1 paquete \textbf{(ping -c 1 \textless maquinaDestino\textgreater)}. \newline
		Interrumpe la captura en pc4 \textbf{(Ctrl+C)}. \newline
		Comprueba el estado de las caches de ARP en pc1, pc4, pc2 y r3 y explica su contenido.
		
		Las caches de ARP en pc1 contiene la dirección IP y Ethernet de pc4, y en pc4 al contrario. En pc2 y en r3 las caches de ARP se encuentran vacías.
	\item Analiza la captura con wireshark. Observa los siguientes campos en los mensajes:
	\begin{itemize}
		\item Mensaje de solicitud de ARP que envia pc1 a pc4.
		\begin{itemize}
			\item Direccion Ethernet destino: ff:ff:ff:ff:ff:ff
			\item Direccion Ethernet origen: ce:42:01:bb:1b:2c
			\item Tipo en la cabecera Ethernet: ARP
			\item Contenido del mensaje de solicitud de ARP: localiza el campo que contiene la direccion IP de la
			maquina sobre la que se esta preguntando su direccion Ethernet.
			
			Target IP address: 200.57.0.40
			
		\end{itemize}
		\item Mensaje de respuesta de ARP que envia pc4 a pc1.
		\begin{itemize}
			\item Dirección Ethernet destino: ce:42:01:bb:1b:2c
			\item Dirección Ethernet origen: 42:d6:24:8c:dc:02
			\item Tipo en la cabecera Ethernet: ARP
			\item Contenido del mensaje de respuesta de ARP: localiza el campo que contiene la dirección Ethernet
			solicitada.
			
			Sender Ethernet address: 42:d6:24:8c:dc:02
		\end{itemize}
		\item  Datagrama IP que envia pc1 a pc4.
		\begin{itemize}
			\item Dirección Ethernet destino: 42:d6:24:8c:dc:02
			\item Dirección Ethernet origen: ce:42:01:bb:1b:2c
			\item Tipo en la cabecera Ethernet: IPv4
			\item Dirección IP origen: 200.57.0.10
			\item Dirección IP destino: 200.57.0.40
			\item Campo TTL: 64
		\end{itemize}
		\item  Datagrama IP que envia pc4 a pc1.
		\begin{itemize}
			\item Dirección Ethernet destino: ce:42:01:bb:1b:2c
			\item Dirección Ethernet origen: 42:d6:24:8c:dc:02
			\item Tipo en la cabecera Ethernet: IPv4
			\item Dirección IP origen: 200.57.0.40
			\item Dirección IP destino: 200.57.0.10
			\item Campo TTL: 64
		\end{itemize}
	\end{itemize}
	NOTA: Si tu captura ha durado mas de 5 segundos, podras ver que, 5 segundos despues de la solictud de
	ARP de pc1 preguntando por pc4, pc4 hace una solicitud de ARP preguntando por pc1, pero esta solicitud de ARP no tiene direccion de destino broadcast, sino la direccion Ethernet de pc1. Esto demuestra que pc4 ya tenia en su cache de ARP la direccion Ethernet de
	pc1 (la aprendio de la solicitud recibida de pc1), y este ARP es simplemente de confirmacion. No todas las implementaciones de TCP/IP realizan esta confirmacion, y en esta asignatura nunca le prestaremos atencion ni preguntaremos sobre ella.
	\item Espera a que la cache de ARP de pc1 este vacia. Ahora 
		vamos a analizar el trafico desde pc1 a pc2. ¿Cuantas capturas de trafico crees que son necesarias para ver todos los paquetes que se generan en el escenario cuando se comunican pc1 y pc2?
		
		Serán necesarias 2 capturas.
		\newline
		Arranca un tcpdump en r3(eth0) guardando la captura en el fichero \color{blue}p3-a-02.cap.\color{black}
		\newline
		Arranca otro tcpdump en pc2 guardando la captura en el fichero \color{blue}p3-a-03.cap.\color{black}
		\newline
		Ejecuta en pc1 un ping a pc2 que envie solo 1 paquete \textbf{(ping -c 1 \textless maquinaDestino\textgreater)}.
		\newline
		Interrumpe las capturas \textbf{(Ctrl+C)}.
		\newline
		Comprueba el estado de las caches de ARP en pc1, pc2, pc4 y r3. Explica su contenido.
		
		Pc1 y Pc2 tiene la dirección de r3, r3 tiene la dirección de ambos, y pc4 está vacia.
	\item Analiza las capturas realizadas con wireshark. Observa en los mensajes los mismo campos que analizaste el apartado anterior. Presta especial atencion a los datagramas IP. Identifica los mensajes en las dos capturas
	que contienen el mismo datagrama IP.  ¿Que direcciones Ethernet tiene la trama que contiene esos datagramas
	en cada captura?  ¿Que valor tiene el campo TTL de la cabecera IP en cada uno?
	
	En la primera captura el primer datagrama Ip tiene la dirección Ethernet de pc1 y la eth0 de r3, y en la segunda captura las direcciones Ip son las de pc2 y la de eth1 de r3.
	
	En el segundo datagrama las direcciones de el primer datagrama están invertidas.
	
	En cuanto al ttl del primer mensaje: en la primera captura es 64 y en la segunda 63; sin embargo para el segundo datagrama es justo al revés, ya que es la respuesta.
\end{enumerate}
\chapter{Configuración de tablas de encaminamiento mediante ficheros de configuración}

Arranca el resto de las máquinas y routers de la figura y obtendrás un diagrama similar a la figura. Ten en
cuenta que en
pc1
ya tendrás configurada una dirección IP, mantén esta configuración.
\newline

\includegraphics*[scale=0.15, center]{enunciado2}
\begin{enumerate}
	\item ¿Cuántas subredes observas en la figura? Escribe la dirección de cada una de estas subredes junto con su
	máscara.
	
	En la figura se observan 4 subredes: 200.0.0.0, 201.0.0.0, 202.0.0.0 y 203.0.0.0 ; todas con la máscara 255.255.255.0
	\item Reinicia las máquinas	pc1, pc2 y pc4.\newline
	Consulta las tablas de encaminamiento en todas las máquinas y
	routers, comprobarás
	que las rutas que
	configuraste en el apartado anterior han desaparecido. Los pcs y
	routers sólo tienen ruta
	hacia las subredes
	a las que están directamente conectados. Por tanto sólo podrán enviar paquetes a sus máquinas vecinas.
	\item Añade rutas en las máquinas adecuadas modificando su fichero /etc/network/interfaces de forma que funcionen las siguientes rutas:
	
	\includegraphics*[width=128mm, scale=0.5, center]{ejercicio4}
	
	\begin{enumerate}[label=\alph*]
		\item Conectividad entre pc1 y pc2 en los dos sentidos, a través de las siguientes rutas:
		\begin{itemize}
			\item pc1 a r3 a pc2			
			\item pc2 a r3 a pc1	
		\end{itemize}
		NOTA: Puedes emplear rutas de máquina, rutas de red o rutas por defecto.
		
		Ejecuta en
		pc1 la orden ping -c 3 \textless dirIPpc2 \textgreater para comprobar si hay conectividad entre las máquinas.
		Para verficar que el camino es el descrito previamente, deberás ejecutar
		route en pc1 y observar qué
		entrada de la tabla de encaminamiento se utiliza cuando se desea alcanzar pc2. Esta entrada debería
		indicar que el siguiente salto es r3. A continuación en r3 deberás ejecutar route y observar qué entrada
		de la tabla de encaminamiento se utiliza cuando se desea alcanzar pc2. Esta entrada debería indicar que
		r3 no necesita ningún router adicional para alcanzar pc2.
		
		De forma análoga deberás consultar las tablas de encaminamiento en el sentido pc2 a r3 a pc1.

		\includegraphics*[width=128mm, scale=0.5, center]{ejercicio6}
		\item  Conectividad entre pc2 y pc3 en los dos sentidos, a través de la siguientes rutas:
		\begin{itemize}
			\item pc2 a r2 a pc3			
			\item pc3 a r2 a pc2	
		\end{itemize}
		Ejecuta en
		pc2 la orden ping -c 3 \textless dirIPpc3 \textgreater para comprobar si hay conectividad entre las máquinas.
		Para verificar que el camino es el descrito previamente, deberás ejecutar
		route en pc2 y observar qué
		entrada
		de la tabla de encaminamiento se utiliza cuando se desea alcanzar pc3. Esta entrada debería indicar que el siguiente salto es r2.
		A continuación en
		r2
		deberás ejecutar
		route y observar qué entrada
		de la tabla de encaminamiento se utiliza cuando se desea alcanzar
		pc3. Esta entrada debería indicar que
		r2
		no necesita ningún router adicional para alcanzar
		pc3.
		
		De forma análoga deberás consultar las tablas de encaminamiento en el sentido pc3 a r2 a pc2.

		\includegraphics*[width=128mm, scale=0.5, center]{ejercicio7}
		\item Conectividad entre pc1 y pc3 en los dos sentidos, a través de las siguientes rutas:
		\begin{itemize}
			\item pc1 a r1 a pc3			
			\item pc3 a r2 a pc1	
		\end{itemize}		
		Ejecuta en
		pc1 la orden ping -c 3 \textless dirIPpc3\textgreater para comprobar si hay conectividad entre las máquinas.
		Para verificar que el camino es el descrito previamente, deberás ejecutar
		route en pc1 y observar qué
		entrada de la tabla de encaminamiento se utiliza cuando se desea alcanzar pc3. Esta entrada debería
		indicar que el siguiente salto es r1. Después, deberás ejecutar route en r1 y observar qué entrada de la
		tabla de encaminamiento se utiliza cuando se desea alcanzar pc3. Esta entrada debería indicar que el
		siguiente salto es r2. A continuación en r2 deberás ejecutar route y observar qué entrada de la tabla de
		encaminamiento se utiliza cuando se desea alcanzar pc3. Esta entrada debería indicar que r2 no necesita
		ningún router adicional para alcanzar pc3.
		
		De forma análoga deberás consultar las tablas de encaminamiento en el sentido pc3 a r2 a r3 a pc1
		
		\includegraphics*[width=128mm, scale=0.5, center]{ejercicio8}
	\end{enumerate}
	\item Ejecuta en
	pc1
	un
	traceroute
	hacia
	pc2
	y en
	pc2
	uno hacia
	pc1
	para comprobar que las rutas son las
	especificadas.
	
	\includegraphics*[width=128mm, scale=0.5, center]{ejercicio9}
	
	\includegraphics*[width=128mm, scale=0.5, center]{ejercicio10}
	\item Ejecuta en
	pc2
	un
	traceroute
	hacia
	pc3
	y en
	pc3
	uno hacia
	pc2
	para comprobar que las rutas son las
	especificadas.
	
	\includegraphics*[width=128mm, scale=0.5, center]{ejercicio11}
	\includegraphics*[width=128mm, scale=0.5, center]{ejercicio12}
	\item Ejecuta en
	pc1
	un
	traceroute hacia pc3 y en pc3 uno
	hacia
	pc1
	para comprobar que las rutas son las
	especificadas. En este último
	traceroute observarás que
	aparecen unos
	*,
	en el final del tema 4 de teoría
	veremos a qué se deben. En la siguiente práctica se estudiarán más en detalle estas situaciones.
	
	\includegraphics*[width=128mm, scale=0.5, center]{ejercicio13}
	\includegraphics*[width=128mm, scale=0.5, center]{ejercicio14}
	\item Antes de continuar asegúrate de que las rutas entre las máquinas son las especificadas en el apartado anterior.
	
	Lanza con
	tcpdump
	capturas en las siguientes máquinas, guardando los ficheros con el nombre que se indica:
	\begin{itemize}
		\item captura en r1(eth0) con nombre de fichero \color{blue} p3-b-01.cap \color{black}
		\item captura en r2(eth0) con nombre de fichero \color{blue} p3-b-02.cap \color{black}
		\item captura en r3(eth1) con nombre de fichero \color{blue} p3-b-03.cap \color{black}
		\item captura en pc3(eth0) con nombre de fichero \color{blue} p3-b-04.cap \color{black}	
	\end{itemize}

	Ejecuta en pc1 un ping a pc3 que envíe sólo 2 paquetes (ping -c 2 \textless máquinaDestino\textgreater).
	
	Interrumpe las 4 capturas (Ctrl+C).
	
	Analiza con wireshark las 4 capturas. Observa en ellas cómo los datagramas IP que se envían y reciben con la orden ping contienen un mensaje de ICMP. Comprueba en estos datagramas:
	
	Captura en r1
	
	\begin{tabular}{|c|c|c|c|c|c|}
		\hline
		Source IP & Destination IP & TTL & Protocolo en IP & Tipo ICMP & Código ICMP \\
		\hline
		200.57.0.10 & 203.57.0.30 & 64 & ICMP & 8 & 0 \\
		\hline 
		203.57.0.30 & 200.57.0.10 & 62 & ICMP & 0 & 0 \\
		\hline
		200.57.0.10 & 203.57.0.30 & 64 & ICMP & 8 & 0 \\
		\hline 
		203.57.0.30 & 200.57.0.10 & 62 & ICMP & 0 & 0 \\
		\hline
	\end{tabular}

	Captura en r2
	
	\begin{tabular}{|c|c|c|c|c|c|}
		\hline
		Source IP & Destination IP & TTL & Protocolo en IP & Tipo ICMP & Código ICMP \\
		\hline
		200.57.0.10 & 203.57.0.30 & 63 & ICMP & 8 & 0 \\
		\hline 
		200.57.0.10 & 203.57.0.30 & 63 & ICMP & 8 & 0 \\
		\hline
	\end{tabular}

	Captura en r3
	
	\begin{tabular}{|c|c|c|c|c|c|}
		\hline
		Source IP & Destination IP & TTL & Protocolo en IP & Tipo ICMP & Código ICMP \\
		\hline
		203.57.0.30 & 200.57.0.10 & 63 & ICMP & 0 & 0 \\
		\hline 
		203.57.0.30 & 200.57.0.10 & 63 & ICMP & 0 & 0 \\
		\hline
	\end{tabular}

	Captura en pc3
	
	\begin{tabular}{|c|c|c|c|c|c|}
		\hline
		Source IP & Destination IP & TTL & Protocolo en IP & Tipo ICMP & Código ICMP \\
		\hline
		200.57.0.10 & 203.57.0.30 & 62 & ICMP & 8 & 0 \\
		\hline 
		203.57.0.30 & 200.57.0.10 & 64 & ICMP & 0 & 0 \\
		\hline
		200.57.0.10 & 203.57.0.30 & 62 & ICMP & 8 & 0 \\
		\hline 
		203.57.0.30 & 200.57.0.10 & 64 & ICMP & 0 & 0 \\
		\hline
	\end{tabular}

	Consultando las capturas, responde a las siguientes cuestiones:
	\begin{enumerate}[label=\alph*]
		\item ¿En qué se distinguen los mensajes "de ida" del ping de los mensajes "de vuelta"?
		
		Se distinguen en el campo type de ICMP, si vale 8 es un mensaje de "ida" y si vale 0 es una respueta.
		\item ¿En qué capturas se pueden ver los mensajes "de ida" del ping? ¿Y los mensajes de vuelta? ¿Por qué?
		
		Los mensajes de "ida" se pueden ver en r1, r2 y pc3, ya que es la ruta de "ida" configurada.
		
		Los mensajes de vuelta se pueden ver en r3 y pc3, ya que es la rura configurada de "vuelta". También se pueden ver en r1, ya que para que el datagrama sea enviado desde r3 a pc1 debe de pasar por un hub, lo que hace que el mensaje llegue.
		\item Comprueba los valores del campo TTL de la cabecera IP de todos los datagramas de todas las capturas
		y explica dichos valores.
		
		Cuando el valor es 64, es el mensaje sin haber pasado por ninguna otra máquina.
		
		Si el valor es 63, el mensaje ha sido reenviado por un router
		
		Y finalmente si el valor es 62, en este caso ya habrá de llegar al destino.
	\end{enumerate}
	\item Arranca de nuevo
	tcpdump en las mismas máquinas e interfaces que lo has hecho anteriormente pero guardando
	las capturas en otros ficheros diferentes:
	\begin{itemize}
		\item captura en r1(eth0) con nombre de fichero \color{blue} p3-b-05.cap \color{black}
		\item captura en r2(eth0) con nombre de fichero \color{blue} p3-b-06.cap \color{black}
		\item captura en r3(eth1) con nombre de fichero \color{blue} p3-b-07.cap \color{black}
		\item captura en pc3(eth0) con nombre de fichero \color{blue} p3-b-08.cap \color{black}	
	\end{itemize}
	Ejecuta en
	pc1
	la orden
	traceroute
	a
	pc3.\newline
	Cuando la orden anterior haya terminado completament, interrumpe las capturas (Ctrl+C).
	A la vista del resultado que se muesta en
	pc1: ¿por qué router
	intermedios ha pasado un paquete para llegar
	de
	pc1
	a
	pc3?
	
	Habrá pasado por r1 y por r2.
	\item Abre con
	wireshark
	los ficheros de captura que has obtenido. Identifica en los ficheros de capturas los
	siguientes paquetes:
	\begin{itemize}
		\item Los 3 mensajes enviados por pc1 con TTL=1
		
		\includegraphics*[width=128mm, scale=0.5, center]{mensajes1}
		\item Los 3 ICMP de TTL excedido enviados por r1
		
		\includegraphics*[width=128mm, scale=0.5, center]{mensajes2}
		\item Los 3 mensajes enviados por pc1 con TTL=2
		
		\includegraphics*[width=128mm, scale=0.5, center]{mensajes3}
		\item Los 3 ICMP de TTL excedido enviados por r2
		
		\includegraphics*[width=128mm, scale=0.5, center]{mensajes4}
		\item Los 3 mensajes enviados por pc1 con TTL=3
		
		\includegraphics*[width=128mm, scale=0.5, center]{mensajes5}
		\item Los 3 ICMP de puerto inalcanzable enviados por pc3
		
		\includegraphics*[width=128mm, scale=0.5, center]{mensajes6}
	\end{itemize}
	\item Consultando las capturas, responde a las siguientes cuestiones:
	\begin{enumerate}[label=\alph*]
		\item ¿Por qué ruta van viajando los mensajes enviados por
		pc1
		con TTL creciente?
		
		Van viajando hacia r1, desde r1 a r2, y de r2 a pc3
		\item ¿Por qué ruta viajan los ICMP enviados por
		r1?
		¿Qué dirección IP usa
		r1
		como IP de origen el enviar
		esos ICMP?
		
		Viajan desde r1 a pc1 usando la dirección IP de r1 eth0 como origen, 200.57.0.1
		\item ¿Por qué ruta viajan los ICMP enviados por
		r2?
		¿Qué dirección IP usa
		r2
		como IP de origen el enviar
		esos ICMP?
		
		Viajan desde r2 a r3, y de r3 a pc1 usando como IP origen 202.57.0.2
		\item ¿Por qué ruta viajan los ICMP enviados por
		pc3?
		
		Viajan desde pc3 a r2, de r2 a r3, y desde r3 a pc1.
		
	\end{enumerate}
\end{enumerate}
\end{document}